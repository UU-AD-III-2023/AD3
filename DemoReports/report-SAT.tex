% !TEX root = demoReport.tex
\newcommand{\p}{\varphi}
\section*{Problem 3: Boolean Satisfiability (SAT)}

\newcommand{\MiniSat}{Mini\-Sat}  % avoids default hyphenation into Min-iSat
\newcommand{\SolverSAT}{\todo{\MiniSat}\xspace}  % or Lingeling, or ...
\newcommand{\TimeoutSAT}{\todo{60.0}}  % timeout, in CPU seconds; MIN 60.0
\newcommand{\green}[1]{{\color{green}#1}}
\newcommand{\gray}[1]{{\color{gray}#1}}


\paragraph{Task~a: Ordered Resolution.}
Consider the following formula in conjunctive normal form (CNF):
\begin{multline*}
  \varphi \equiv (x_1 \lor x_2) \land
  (x_3 \lor x_4) \land
  (x_5 \lor x_6) \land
  (\neg x_1 \lor \neg x_3) \land
  (\neg x_1 \lor \neg x_5) \\ \land
  (\neg x_3 \lor \neg x_5) \land
  (\neg x_2 \lor \neg x_4) \land
  (\neg x_2 \lor \neg x_6) \land
  (\neg x_4 \lor \neg x_6)
\end{multline*}
We apply resolution by selecting the variables in their own given order. In blue are the new clauses from the resolution.
\begin{align*}
\varphi \equiv & (x_1 \lor x_2) \land
  	(x_3 \lor x_4) \land
  	(x_5 \lor x_6) \land
  	(\neg x_1 \lor \neg x_3) \land
  	(\neg x_1 \lor \neg x_5) \\ & \land
  	(\neg x_3 \lor \neg x_5) \land
  	(\neg x_2 \lor \neg x_4) \land
  	(\neg x_2 \lor \neg x_6) \land
  	(\neg x_4 \lor \neg x_6) \\
  \equiv & \todo{(x_2 \lor \neg x_3)} \land
  	\todo{(x_2 \lor \neg x_5)} \land
  	(x_3 \lor x_4) \land
	(x_5 \lor x_6) \land
	(\neg x_3 \lor \neg x_5) \\ & \land
	(\neg x_2 \lor \neg x_4) \land
  	(\neg x_2 \lor \neg x_6) \land
  	(\neg x_4 \lor \neg x_6) \\
  \equiv & \todo{(\neg x_3 \lor \neg x_4)} \land
  	\todo{(\neg x_3 \lor \neg x_6)} \land
	\todo{(\neg x_5 \lor \neg x_4)} \land
	\todo{(\neg x_5 \lor \neg x_6)} \\ & \land
	(x_3 \lor x_4) \land
	(x_5 \lor x_6) \land
	(\neg x_3 \lor \neg x_5) \land
  	(\neg x_4 \lor \neg x_6) \\
  \equiv & \todo{(x_4 \lor \neg x_6)} \land
  	\todo{(x_4 \lor \neg x_5)} \land
    \todo{(\neg x_4 \lor \neg x_5)} \land
  	\todo{(\neg x_5 \lor \neg x_6)} \\ & \land
  	(x_5 \lor x_6) \land
    (\neg x_4 \lor \neg x_6) \\
  \equiv & \todo{(\neg x_5 \lor \neg x_6)} \land
    \todo{(\neg x_6)} \land
    \todo{(\neg x_5)} \land
    (x_5 \lor x_6) \\
    \equiv & (\neg x_6) \land \todo{(x_6)} \\
    \equiv & \todo{()}
\end{align*}
As we reach the empty clause, the formula is unsatisfiable.

%\todo{Perform ordered resolution
%  % (cf.\ \emph{Introduction to SAT}, slide~19)
%  on this formula, selecting the variables in the order given by their
%  index (i.e., $x_1$ before $x_2$ before~\dots).  Show the result
%  after each iteration.
%  %
%  Based on your resolution, is~$\varphi$ satisfiable?
%}

\paragraph{Task~b: DPLL.}
Consider again the formula~$\varphi$ given in Task~a.
%
On the initial formula $\varphi$ we have no unit clause and no pure literal, so we select the variable $x_1$ and run $\textsf{DPLL}(\varphi \land x_1)$: \\
The formula we now apply \textsf{DPLL} is now:
\begin{align*}
\p_1 \equiv \p \land x_1 \equiv & (x_1 \lor x_2) \land
  (x_3 \lor x_4) \land
  (x_5 \lor x_6) \land
  (\neg x_1 \lor \neg x_3) \land
  (\neg x_1 \lor \neg x_5) \\ & \land
  (\neg x_3 \lor \neg x_5) \land
  (\neg x_2 \lor \neg x_4) \land
  (\neg x_2 \lor \neg x_6) \land
  (\neg x_4 \lor \neg x_6) \land (x_1)
\end{align*}
We have a unit clause now and make unit propagation with $x_1$
\begin{align*}
\p_2 \equiv &
  (x_3 \lor x_4) \land
  (x_5 \lor x_6) \land
  (\neg x_3) \land
  (\neg x_5) \\ & \land
  (\neg x_3 \lor \neg x_5) \land
  (\neg x_2 \lor \neg x_4) \land
  (\neg x_2 \lor \neg x_6) \land
  (\neg x_4 \lor \neg x_6)
\end{align*}
We have new unit clauses: First we apply the unit propagation on $\neg x_3$ and get:
\begin{align*}
 \p_3 \equiv &
   (x_5 \lor x_6) \land
   (\neg x_5) \\ & \land
   (\neg x_2) \land
   (\neg x_2 \lor \neg x_6) \land
   (\neg x_4 \lor \neg x_6)
\end{align*}
We apply unit propagation on $x_4$ and get:
\begin{align*}
 \p_4 \equiv &
   (x_5 \lor x_6) \land
   (\neg x_5) \\ & \land
   (\neg x_2) \land
   (\neg x_2 \lor \neg x_6) \land
   (\neg x_6)
\end{align*}
We apply unit propagation $\neg x_5$ and get:
\begin{align*}
 \p_5 \equiv &
   (x_6) \land
   (\neg x_2) \\ & \land
   (\neg x_2 \lor \neg x_6) \land
   (\neg x_6)
\end{align*}
We apply unit propagation to $x_6$ and get:
\begin{align*}
 \p_6 \equiv &
   (\neg x_2) \land
   (\neg x_2) \land
   ()
\end{align*}
and get an empty clause. We make backtracking and run $\mathsf{DPLL}(\p \land \neg x_1)$. We make unit propagation with $\neg x_1$:
\begin{align*}
\p_7 \equiv \p \land \neg x_1 \equiv & (x_2) \land
  (x_3 \lor x_4) \land
  (x_5 \lor x_6) \land
  (\neg x_3 \lor \neg x_5) \\ & \land
  (\neg x_2 \lor \neg x_4) \land
  (\neg x_2 \lor \neg x_6) \land
  (\neg x_4 \lor \neg x_6)
\end{align*}
We make unit propagation with $x_2$ and get:
\begin{align*}
\p_8 \equiv &
  (x_3 \lor x_4) \land
  (x_5 \lor x_6) \land
  (\neg x_3 \lor \neg x_5) \\ & \land
  (\neg x_4) \land
  (\neg x_6) \land
  (\neg x_4 \lor \neg x_6)
\end{align*}
We make unit propagation with $\neg x_4$ and get:
\begin{align*}
\p_9 \equiv &
  (x_3) \land
  (x_5 \lor x_6) \\ & \land
  (\neg x_3 \lor \neg x_5)  \land
  (\neg x_6)
\end{align*}
We we make unit propagation with $x_3$ and get:
\begin{align*}
\p_{10} \equiv &
  (x_5 \lor x_6) \land
  (\neg x_5)  \land
  (\neg x_6)
\end{align*}
We make unit propagation with $\neg x_5$ and get:
\begin{align*}
\p_{11} \equiv &
  (x_6) \land
  (\neg x_6)
\end{align*}
We make unti propagation with $x_6$ and get:
\begin{align*}
  \p_{12} \equiv ()
\end{align*}
Hence we get $\p = \mathsf{UNSAT}$.

%\todo{Explain in detail how the DPLL algorithm,
%  % (cf.\ \emph{Introduction to SAT}, slide~26),
%  when applied to~$\varphi$, determines whether the formula is
%  satisfiable.
%  %
%  Assume that the variables are selected in the order given by their
%  index (i.e., $x_1$ before $x_2$ before~\dots), and that they are
%  assigned~$1$ (i.e., True) before they are assigned~$0$ (i.e.,
%  False).
%  %
%  Remember to perform unit propagation and to apply the pure-literal
%  rule where possible, in order to prune parts of the search space.
%}

\paragraph{Task~c: CDCL.}
Consider the following CNF formula:
\[
  (x_1 \lor x_8 \lor \neg x_2) \land
  (x_1 \lor \neg x_3) \land
  (x_2 \lor x_3 \lor x_4) \land
  (\neg x_4 \lor \neg x_5) \land
  (x_7 \lor \neg x_4 \lor \neg x_6) \land
  (x_5 \lor x_6)
\]
Assume that~$x_7$ has been assigned~$0$ at decision level~$2$, and
that~$x_8$ has been assigned~$0$ at decision level~$3$.
%
Moreover, assume that the current decision assignment is $x_1 = 0$ at
decision level~$5$.
%
\begin{center}
\begin{tikzpicture}
  \node (2) [state] {$2$};
  \node (1) [above = of 2] {$\vdots$};
  \node (a2) [left = of 2] {$x_7 \mapsto 0$};
  \node (3) [state, below = of 2] {$3$};
  \node (a3) [left = of 3] {$x_8 \mapsto 0$};
  \node (4) [below = of 3] {$\vdots$};
  \node (5) [state, below = of 4] {$5$};
  \node (a5) [left = of 5] {$x_1 \mapsto 0$};
  \path [-stealth, thick]
  		(2) edge node {} (3)
		(3) edge node {} (4)
		(4) edge node {} (5);
\end{tikzpicture}
\end{center}
Using \green{green} text for trustified literals and \gray{gray} for gray for false ones we get the formula:
\begin{align*}
  (\gray{x_1} \lor \gray{x_8} \lor \neg x_2) \land
  (\gray{x_1} \lor \neg x_3) \land
  (x_2 \lor x_3 \lor x_4) \land
  (\neg x_4 \lor \neg x_5) \land
  (\gray{x_7} \lor \neg x_4 \lor \neg x_6) \land
  (x_5 \lor x_6)
\end{align*}


%\todo{Draw a resulting implication graph.
%  % (cf.\ \emph{Mini-tutorial on conflict-driven clause learning
%  % solvers}, slide~4).
%  Does the graph contain any conflicts?  If so, then mark these
%  clearly, and provide a conflict clause.
%}


\paragraph{Task~d: Encoding.}~ \todo{Describe your encoding, citing
  either~\cite{AMK:CNF:Sinz}, or~\cite[Section~2.2.2]{CNFencodings},
  or both, if you use their ideas: first explain the meaning of the
  Boolean variables that you use in your formula~$\varphi_{d,c,e}$;
  then explain the encodings by the help functions
  % $\textsc{AndImply}(x_1,\dots,x_n,b)$ and
  % $\textsc{OrImply}(x_1,\dots,x_n,b)$
  of the hint that you actually use (there is no need to explain
  $\textsc{AtMost}(k,x_1,\dots,x_n)$ if you use~\cite{AMK:CNF:Sinz});
  and finally explain how the constraints of the problem are encoded
  using those variables and help functions.}

We chose the programming language \todo{\filler}, for which a compiler
or interpreter is available on the Linux computers of the IT
department.  All source code is \todo{uploaded} with this report.  The
compilation and running instructions are \todo{\filler}.

We validated the correctness of our encoding and implementation by
\todo{checking its outputs on many instances via the provided
  polynomial-time solution checker}.

\paragraph{Task~e: Experiments.}
We chose the SAT solver \SolverSAT for our experiments.
%
We \todo{used or did not use} the provided script for running the
experiments and tabling their results
% specification of the ThinLinc machines of the IT department, or
% replace by a similar-looking specification of your own hardware:
\todo{under Linux Ubuntu~18.04 ($64$~bit) on an Intel Xeon E5520 of
  $2.27$~GHz, with $4$~processors of $4$~cores each, with a $70$~GB
  RAM and an $8$~MB L3 cache (a ThinLinc computer of the IT
  department)}.
% on the AD3 web interface to \MiniSat on a $2.69$~GHz Intel
% Core i7, with $8$~GB RAM and $4$~MB L3 cache.

The results are in Table~\ref{tab:res:sat}.
%
The trivially unsatisfiable instances (which are the ones that violate
the inequality $e \leq \Floor{\frac{d-1}{c-1}}$) that \todo{were
  actually attempted in our experiments} are
%% In other words: delete the ones that are NOT in YOUR table, and add
%% the missing ones that ARE in YOUR table:
$\Tuple{8,2,8}$, $\Tuple{10,2,10}$, $\Tuple{12,2,12}$,
$\Tuple{14,2,14}$, $\Tuple{16,2,16}$, $\Tuple{15,3,8}$,
\todo{\filler}, $\Tuple{16,4,6}$, \todo{\filler}, and \todo{\filler}.
%
We observe that our encoding detects their trivial unsatisfiability in
\todo{\filler} time.

\begin{table}[t]  % make it float to the top of a page
  \centering
  \input{results-SAT.tex}
  \caption{Cruise design: satisfiability and runtime (in seconds)
    using \SolverSAT, with a timeout of $\TimeoutSAT$~CPU seconds; a
    timeout is denoted by~`t/o'; if no timeout occurred, then proven
    satisfiability is denoted by `sat' and proven unsatisfiability by
    `unsat', else trivial unsatisfiability is denoted by `unsat' and
    the unknown status is denoted by~`?'.}
  \label{tab:res:sat}
\end{table}

%%% Local Variables:
%%% mode: latex
%%% TeX-master: "demoReport"
%%% End:
