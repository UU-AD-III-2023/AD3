\section*{Problem 3: Boolean Satisfiability (SAT)}

\newcommand{\MiniSat}{Mini\-Sat}  % avoids default hyphenation into Min-iSat
\newcommand{\SolverSAT}{\todo{\MiniSat}\xspace}  % or Lingeling, or ...
\newcommand{\TimeoutSAT}{\todo{60.0}}  % timeout, in CPU seconds; MIN 60.0

\paragraph{Task~a: Ordered Resolution.}
Consider the following formula in conjunctive normal form (CNF):
\begin{multline*}
  \varphi \equiv (x_1 \lor x_2) \land
  (x_3 \lor x_4) \land
  (x_5 \lor x_6) \land
  (\neg x_1 \lor \neg x_3) \land
  (\neg x_1 \lor \neg x_5) \\ \land
  (\neg x_3 \lor \neg x_5) \land
  (\neg x_2 \lor \neg x_4) \land
  (\neg x_2 \lor \neg x_6) \land
  (\neg x_4 \lor \neg x_6)
\end{multline*}
\todo{Perform ordered resolution
  % (cf.\ \emph{Introduction to SAT}, slide~19)
  on this formula, selecting the variables in the order given by their
  index (i.e., $x_1$ before $x_2$ before~\dots).  Show the result
  after each iteration.
  % 
  Based on your resolution, is~$\varphi$ satisfiable?
}

\paragraph{Task~b: DPLL.}
Consider again the formula~$\varphi$ given in Task~a.
% 
\todo{Explain in detail how the DPLL algorithm,
  % (cf.\ \emph{Introduction to SAT}, slide~26),
  when applied to~$\varphi$, determines whether the formula is
  satisfiable.
  % 
  Assume that the variables are selected in the order given by their
  index (i.e., $x_1$ before $x_2$ before~\dots), and that they are
  assigned~$1$ (i.e., True) before they are assigned~$0$ (i.e.,
  False).
  % 
  Remember to perform unit propagation and to apply the pure-literal
  rule where possible, in order to prune parts of the search space.
}

\paragraph{Task~c: CDCL.}
Consider the following CNF formula:
\[
  (x_1 \lor x_8 \lor \neg x_2) \land
  (x_1 \lor \neg x_3) \land
  (x_2 \lor x_3 \lor x_4) \land
  (\neg x_4 \lor \neg x_5) \land
  (x_7 \lor \neg x_4 \lor \neg x_6) \land
  (x_5 \lor x_6)
\]
Assume that~$x_7$ has been assigned~$0$ at decision level~$2$, and
that~$x_8$ has been assigned~$0$ at decision level~$3$.
% 
Moreover, assume that the current decision assignment is $x_1 = 0$ at
decision level~$5$.
% 
\todo{Draw a resulting implication graph.
  % (cf.\ \emph{Mini-tutorial on conflict-driven clause learning
  % solvers}, slide~4).
  Does the graph contain any conflicts?  If so, then mark these
  clearly, and provide a conflict clause.
}

\paragraph{Task~d: Encoding.}~ \todo{Describe your encoding, citing
  either~\cite{AMK:CNF:Sinz}, or~\cite[Section~2.2.2]{CNFencodings},
  or both, if you use their ideas: first explain the meaning of the
  Boolean variables that you use in your formula~$\varphi_{d,c,e}$;
  then explain the encodings by the help functions
  % $\textsc{AndImply}(x_1,\dots,x_n,b)$ and
  % $\textsc{OrImply}(x_1,\dots,x_n,b)$
  of the hint that you actually use (there is no need to explain
  $\textsc{AtMost}(k,x_1,\dots,x_n)$ if you use~\cite{AMK:CNF:Sinz});
  and finally explain how the constraints of the problem are encoded
  using those variables and help functions.}

We chose the programming language \todo{\filler}, for which a compiler
or interpreter is available on the Linux computers of the IT
department.  All source code is \todo{uploaded} with this report.  The
compilation and running instructions are \todo{\filler}.

We validated the correctness of our encoding and implementation by
\todo{checking its outputs on many instances via the provided
  polynomial-time solution checker}.

\paragraph{Task~e: Experiments.}
We chose the SAT solver \SolverSAT for our experiments.
%
We \todo{used or did not use} the provided script for running the
experiments and tabling their results
% specification of the ThinLinc machines of the IT department, or
% replace by a similar-looking specification of your own hardware:
\todo{under Linux Ubuntu~18.04 ($64$~bit) on an Intel Xeon E5520 of
  $2.27$~GHz, with $4$~processors of $4$~cores each, with a $70$~GB
  RAM and an $8$~MB L3 cache (a ThinLinc computer of the IT
  department)}.
% on the AD3 web interface to \MiniSat on a $2.69$~GHz Intel
% Core i7, with $8$~GB RAM and $4$~MB L3 cache.

The results are in Table~\ref{tab:res:sat}.
%
The trivially unsatisfiable instances (which are the ones that violate
the inequality $e \leq \Floor{\frac{d-1}{c-1}}$) that \todo{were
  actually attempted in our experiments} are
%% In other words: delete the ones that are NOT in YOUR table, and add
%% the missing ones that ARE in YOUR table:
$\Tuple{8,2,8}$, $\Tuple{10,2,10}$, $\Tuple{12,2,12}$,
$\Tuple{14,2,14}$, $\Tuple{16,2,16}$, $\Tuple{15,3,8}$,
\todo{\filler}, $\Tuple{16,4,6}$, \todo{\filler}, and \todo{\filler}.
%
We observe that our encoding detects their trivial unsatisfiability in
\todo{\filler} time.

\begin{table}[t]  % make it float to the top of a page
  \centering
  \begin{tabular}{rrrrr}  % right alignment --> decimal point alignment
$d$ & $c$ & $e$ & status & time \\
\midrule
 8 & 2 &  7 &   sat &  \\
 8 & 2 &  8 & unsat &  \\ % trivially
10 & 2 &  9 &   sat &  \\
10 & 2 & 10 & unsat &  \\ % trivially
12 & 2 & 11 &   sat &  \\
12 & 2 & 12 & unsat &  \\ % trivially
14 & 2 & 12 &   sat &  \\
14 & 2 & 13 &   sat &  \\
14 & 2 & 14 & unsat &  \\ % trivially
16 & 2 & 10 &   sat &  \\
16 & 2 & 11 &   sat &  \\
16 & 2 & 12 &   sat &  \\
16 & 2 & 13 &   sat &  \\
16 & 2 & 14 &   sat &  \\
16 & 2 & 15 &   sat &  \\
16 & 2 & 16 & unsat &  \\ % trivially
\end{tabular}
~~
\begin{tabular}{rrrrr}
$d$ & $c$ & $e$ & status & time \\
\midrule
12 & 3 &  4 &   sat &  \\
12 & 3 &  5 & unsat &  \\ % not trivially!
15 & 3 &  6 &   sat &  \\
15 & 3 &  7 &   sat &  \\
15 & 3 &  8 & unsat &  \\ % trivially
18 & 3 &  6 &   sat &  \\
18 & 3 &  7 &   sat &  \\
18 & 3 &  8 &     ? &  \\
21 & 3 &  6 &   sat &  \\
21 & 3 &  7 &   sat &  \\
21 & 3 &  8 &   sat &  \\
21 & 3 &  9 &     ? &  \\
24 & 3 &  6 &   sat &  \\
24 & 3 & \dots& sat &  \\
24 & 3 &  9 &   sat &  \\
24 & 3 & 10 &     ? &  \\
\end{tabular}
~~
\begin{tabular}{rrrrr}
$d$ & $c$ & $e$ & status & time \\
\midrule
16 & 4 &  5 &   sat &  \\
16 & 4 &  6 & unsat &  \\ % trivially
20 & 4 &  4 &   sat &  \\
20 & 4 &  5 &   sat &  \\
20 & 4 &  6 &     ? &  \\
24 & 4 &  4 &   sat &  \\
24 & 4 &  5 &   sat &  \\
24 & 4 &  6 &     ? &  \\
28 & 4 &  4 &   sat &  \\
28 & 4 &  5 &   sat &  \\
28 & 4 &  6 &   sat &  \\
28 & 4 &  7 &     ? &  \\
32 & 4 &  3 &   sat &  \\
32 & 4 & \dots& sat &  \\
32 & 4 &  9 &   sat &  \\
32 & 4 & 10 &   sat &  
\end{tabular}

  \caption{Cruise design: satisfiability and runtime (in seconds)
    using \SolverSAT, with a timeout of $\TimeoutSAT$~CPU seconds; a
    timeout is denoted by~`t/o'; if no timeout occurred, then proven
    satisfiability is denoted by `sat' and proven unsatisfiability by
    `unsat', else trivial unsatisfiability is denoted by `unsat' and
    the unknown status is denoted by~`?'.}
  \label{tab:res:sat}
\end{table}

%%% Local Variables:
%%% mode: latex
%%% TeX-master: "demoReport"
%%% End:
