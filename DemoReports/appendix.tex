% !TEX root = demoReport.tex
\clearpage
%\section*{More \LaTeX\ and Technical Writing Advice}
%
%Unnumbered itemisation (only to be used when the order of the items
%does \emph{not} matter):\footnote{Use footnotes very sparingly, and
%  note that footnote pointers are \emph{never} preceded by a space and
%  always glued immediately \emph{behind} the punctuation, if there is
%  any.}
%\begin{itemize}
%\item Unnumbered displayed formula:
%  \[
%  E = m \cdot c^2
%  \]
%\item Numbered displayed formula, which is cross-referenced somewhere:
%  \begin{equation}
%    \label{eq:emc2}
%    E = m \cdot c^2
%  \end{equation}
%\item Formula --- the same as formula~(\ref{eq:emc2}) --- spanning
%  more than one line:
%  \begin{gather*}
%    E \\ = m \cdot c^2
%  \end{gather*}  
%\end{itemize}
%Numbered itemisation (only to be used when the order of the items
%\emph{does} matter):
%\begin{enumerate}
%\item First do this.
%\item\label{item:that} Then do that.
%\item If we are not finished, then go back to Step~\ref{item:that},
%  else stop.
%\end{enumerate}
%
%Tables and elementary mathematics are typeset as exemplified in
%Table~\ref{tab:maths}; see
%\url{http://tug.ctan.org/info/short-math-guide/short-math-guide.pdf}
%for many more details.
%
%\begin{table}[t] % make it float to the top of a page
%  \centering
%  \begin{tabular}{rlc} % right left centre
%    \toprule
%    Topic & \LaTeX\ code & Appearance \\
%    \midrule
%    Greek letter & \verb|$\Theta,\Omega,\epsilon$| & $\Theta,\Omega,\epsilon$ \\
%    multiplication & \verb|$m \cdot n$| & $m \cdot n$ \\
%    division & \verb|$\frac{m}{n}, m \div n$| & $\frac{m}{n}, m \div n$ \\
%    rounding down & \verb|$\left\lfloor n \right\rfloor$| & $\left\lfloor n \right\rfloor$ \\
%    rounding up & \verb|$\left\lceil n \right\rceil$| & $\left\lceil n \right\rceil$ \\
%    binary modulus & \verb|$m \bmod n$| & $m \bmod n$ \\
%    unary modulus & \verb|$m \equiv n \mod \ell$| & $m \equiv n \mod \ell$ \\
%    root & \verb|$\sqrt{n},\sqrt[3]{n}$| & $\sqrt{n},\sqrt[3]{n}$ \\
%    exponentiation, superscript & \verb|$n^{i}$| & $n^{i}$ \\
%    subscript & \verb|$n_{i}$| & $n_{i}$ \\
%    overline & \verb|$\overline{n}$| & $\overline{n}$ \\
%    base $2$ logarithm & \verb|$\lg n$| & $\lg n$ \\
%    base $b$ logarithm & \verb|$\log_b n$| & $\log_b n$ \\
%    binomial & \verb|$\binom{n}{k}$| & $\binom{n}{k}$ \\
%    sum & \verb|\[\sum_{i=1}^n i\]| & $\displaystyle\sum_{i=1}^n i$ \\
%    numeric comparison & \verb|$\leq,<,=,\neq,>,\geq$| & $\leq,<,=,\neq,>,\geq$ \\
%    non-numeric comparison & \verb|$\prec,\nprec,\preceq,\succeq$| & $\prec,\nprec,\preceq,\succeq$ \\
%    extremum & \verb|$\min,\max,+\infty,\bot,\top$| & $\min,\max,+\infty,\bot,\top$ \\
%    function & \verb|$f\colon A\to B,\circ,\mapsto$| & $f\colon A\to B,\circ,\mapsto$ \\
%    sequence, tuple & \verb|$\langle a,b,c \rangle$| & $\langle a,b,c \rangle$ \\
%    set & \verb|$\{a,b,c\},\emptyset,\mathbb{N}$| & $\{a,b,c\},\emptyset,\mathbb{N}$ \\
%    set membership & \verb|$\in,\not\in$| & $\in,\not\in$ \\
%    set comprehension & \verb|$\{i \mid 1 \leq i \leq n\}$| & $\{i \mid 1 \leq i \leq n\}$ \\
%    set operation & \verb|$\cup,\cap,\setminus,\times$| & $\cup,\cap,\setminus,\times$ \\
%    set comparison & \verb|$\subset,\subseteq,\not\supset$| & $\subset,\subseteq,\not\supset$ \\
%    logic quantifier & \verb|$\forall,\exists,\nexists$| & $\forall,\exists,\nexists$ \\
%    logic connective & \verb|$\land,\lor,\neg,\Rightarrow$| & $\land,\lor,\neg,\Rightarrow$ \\
%    logic & \verb|$\models,\equiv,\vdash$| & $\models,\equiv,\vdash$ \\
%    miscellaneous & \verb|$\&,\#,\approx,\sim,\ell$| & $\&,\#,\approx,\sim,\ell$ \\
%    dots & \verb|$\ldots,\cdots,\vdots,\ddots$| & $\ldots,\cdots,\vdots,\ddots$ \\
%    dots (context-sensitive) & \verb|$1,\dots,n; 1+\dots+n$| & $1,\dots,n; 1+\dots+n$ \\
%    parentheses (autosizing) & \verb|$\left(m^{n^k}\right),(m^{n^k})$| & $\left(m^{n^k}\right),(m^{n^k})$ \\
%    identifier of $>1$ character & \verb|$\mathit{identifier}$| & $\textit{identifier}$ \\
%    hyphen, \emph{n}-dash, \emph{m}-dash, minus & \verb|-|, \verb|--|, \verb|---|, \verb|$-$| & -, --, ---, $-$ \\
%    \bottomrule
%  \end{tabular}
%  \caption{The typesetting of elementary mathematics.  Note very carefully
%    when italics are used by \LaTeX\ and when not, as well as all the
%    horizontal and vertical spacing performed by \LaTeX.}
%  \label{tab:maths}
%\end{table}
%
%Use \verb|\mathit{...}| in mathematical mode for each multiple-letter
%identifier in order to avoid typesetting the identifier like the
%product of single-letter ones.  For example, note the typographic
%difference between the identifier $\mathit{WL}$, obtained through
%\verb|$\mathit{WL}$|, and the product $WL$, where there is a small
%space between the $W$ and the $L$, obtained through \verb|$WL$|.
%
%Do \emph{not} use programming-language-style lower-ASCII notation
%(such as $!$ for negation, $\&\&$ for conjunction, $||$ for
%disjunction, and the equality sign $=$ for assignment) in algorithms
%or formulas (but rather use $\neg$ or $\mathbf{not}$, $\land$ or $\&$
%or $\mathbf{and}$, $\lor$ or $\mathbf{or}$, and $\gets$ or
%$\coloneqq$, respectively), as this testifies to a very strong
%confusion of concepts.
%
%Figures can be imported with \verb|\includegraphics|
%% (such as Figure~\ref{fig:demo})
%or drawn inside the \LaTeX\ source code using the highly declarative
%notation of the \texttt{tikz} package: see Figure~\ref{fig:trees} for
%sample drawings.  It is perfectly acceptable in this course to include
%scans or photos of drawings that were carefully done by hand.
%
%%\begin{figure}[t] % make it float towards the top of a page
%%  \centering
%%  \includegraphics[height=5cm]{lulu.jpg}
%%  \caption{The text under the figure}
%%  \label{fig:demo}
%%\end{figure}
%
%\begin{figure}[t] % make it float to the top of a page
%  \begin{center}
%    \begin{tikzpicture}
%      [level 1/.style={sibling distance=30mm},
%       level 2/.style={sibling distance=15mm},
%       level 2/.style={sibling distance=10mm}]
%      \tikzstyle{every node}=[circle,draw]
%      \node{3}
%      child{
%        node{1}
%        child{node{0}}
%        child{node{2}}
%      }
%      child{
%        node{7}
%        child{
%          node{5}
%          child{node{4}}
%          child{node{6}}
%        }
%        child{node{8}}
%      };
%    \end{tikzpicture} \hspace{4mm}
%    \begin{tikzpicture}
%      [level 1/.style={sibling distance=30mm},
%       level 2/.style={sibling distance=15mm}]
%      \tikzstyle{every node}=[circle,draw]
%      \node{1}
%      child{
%        node{2}
%        child{node{8}}
%        child{node{6}}
%      }
%      child{
%        node{1}
%        child{node{6}}
%        child[missing]{node{k}}
%      }
%      ;
%    \end{tikzpicture} \hspace{4mm}
%    \begin{tikzpicture}[grow via three points={%
%        one child at (0,-1.5) and two children at (0,-1.5) and (-1.5,-1.5)}]
%      \tikzstyle{every node}=[circle,draw]
%      \node at (0,0) {6}
%      child{node{29}}
%      child{
%        node{14}
%        child{
%          node{38}
%        }
%      }
%      child{
%        node{8}
%        child{node{17}}
%        child{
%          node{11}
%          child{node{27}}
%        }
%      }
%      ;
%    \end{tikzpicture}
%  \end{center}
%  \caption{A binary search tree (on the left), a binary min-heap (in
%    the middle), and a binomial tree of rank $3$ (on the right).}
%  \label{fig:trees}
%\end{figure}
%
%If you are not sure whether you will stick to your current choice of
%notation or terminology, then introduce a new (possibly parametric)
%command.  For example, upon
%\begin{center}
%  \verb|\newcommand{\Cardinality}[1]{\left\lvert#1\right\rvert}|
%\end{center}
%the formula \verb|$\Cardinality{S}$| typesets the cardinality of set
%$S$ as $\Cardinality{S}$ with autosized vertical bars and proper
%spacing, but upon changing the definition of that parametric command
%to
%\begin{center}
%  \verb|\newcommand{\Cardinality}[1]{\# #1}|
%\end{center}
%and recompiling, the formula \verb|$\Cardinality{S}$| typesets the
%cardinality of set $S$ as $\#S$.
%%
%You can thus obtain an arbitrary number of changes in the document
%with a \emph{constant}-time change in its source code, rather than
%having to perform a \emph{linear}-time find-and-replace operation
%within the source code, which is painstaking and error-prone.  The
%source code of this document has some useful predefined commands about
%mathematics and algorithms.
%
%Use commands on positioning (such as \verb|\hspace|, \verb|\vspace|,
%and \verb|\noindent|) and appearance (such as \verb|\small| for
%reducing the font size, and \verb|\textit| for italics) very
%sparingly, and ideally only in (parametric) commands, as the very idea
%of mark-up languages such as \LaTeX\ is to let the class designer
%(usually a trained professional typesetter) decide on where things
%appear and how they look.  For example, \verb|\emph| (for emphasis)
%compiles (outside italicised environments, such as \texttt{theorem})
%into \textit{italics} under the \texttt{article} class used for this
%document, but it may compile into \textbf{boldface} under some other
%class.
%\begin{center}
%  \textbf{If you do not (need to) worry about \emph{how} things look, \\
%    then you can fully focus on \emph{what} you are trying to
%    express!}
%\end{center}
%
%Note that \emph{no} absolute numbers are used in the \LaTeX\ source
%code for any of the references inside this document.  For ease of
%maintenance, \verb|\label| is used for giving a label to something
%that is automatically numbered (such as an algorithm, equation,
%figure, footnote, item, line, part, section, subsection, or table),
%and \verb|\ref| is used for referring to a label.  An item in the
%bibliography file is referred to by \verb|\cite| instead.  Upon
%changing the text, it suffices to recompile, once or twice, and
%possibly to run BibTeX again, in order to update all references
%consistently.
%
%Always write
%%
%\verb| Table|$\sim$\verb|\ref{tab:maths} |
%%
%instead of
%%
%\verb| Table \ref{tab:maths}|,
%%
%by using the non-breaking space (which is typeset as the tilde $\sim$)
%instead of the normal space, because this avoids that a
%cross-reference is spread across a line break, as for example in
%``Table \ref{tab:maths}'', which is considered poor typesetting.
%
%The rules of English for how many spaces to use before and after
%various symbols are given in Table~\ref{tab:spacing}.  Beware that
%they may be very different from the rules in your native language.
%
%\begin{table}[t]
%  \centering
%  \begin{tabular}{cc|c|c}
%    \toprule
%    \multicolumn{2}{c}{} & \multicolumn{2}{l}{number of spaces after} \\
%    \cmidrule{3-4}
%    \multicolumn{2}{c}{} & 0 & 1 \\
%    \midrule
%    \multirow{2}{*}{number of spaces before} & 0 & / - & , : ; . ! ?
%    ) ] \} ' '' \% \\
%    \cmidrule{2-4}
%    & 1 & ( [ \{ ` `` & -- (\emph{n}-dash) --- (\emph{m}-dash) \\
%    \bottomrule
%  \end{tabular}
%  \caption{Spacing rules of English}
%  \label{tab:spacing}
%\end{table}
%
%\vfill
%
%\noindent
%\handpoint\ Feel free to report to the head teacher any other features
%that you would have liked to see discussed and exemplified in this
%template document.
